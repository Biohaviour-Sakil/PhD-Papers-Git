\section{Related Work}
\label{sec:related_work}
%\noindent In recent years, researchers have proposed various anomaly-based IDSs for Cloud data centres, which implement machine learning algorithms. We discuss different types of machine learning algorithms used for Cloud anomaly detection as follows. 
\noindent In recent years, researchers have proposed various anomaly detection systems for Cloud data centres. We classify them based on the machine learning algorithms which they implement.

%\subsection {Cloud Intrusion Detection Systems (IDSs)}
\textit{(i) Supervised learning algorithms.}
Supervised learning algorithms rely on labelled training data to detect previously known anomalies. 
%Gupta and Kumar~\cite{Gupta:%2015:ISC:2738886.2738910} suggest an Immediate System Call pattern detection in order to track patterns in the system call logs which can identify a malicious program execution inside VMs. 
%In this case, the detection is not in real-time and the detection is only effective when there is a previously generated pattern of the immediate system calls of the monitored programs.
Li et al.~\cite{ml_based:2012} propose an Artificial Neural Network (ANN) based intrusion detection system for Cloud. The ANN algorithm learns the ``normal" and the ``anomalous" behaviour from a large dataset of VM network traffic. The learned ANN is capable of detecting Cloud security attacks with accurate results.
An anomaly detection system suitable for the hypervisor layer is proposed in~\cite{Pandeeswari2016}. The anomaly detection in this case is based on a mixture of Fuzzy C-Means clustering algorithm and Artificial Neural Network (FCM-ANN) which results in better accuracy and lower false positive rate than the classic ANN and Naive Bayes classifier for detecting various Cloud security attacks. 
The authors in~\cite{supervised_ml:2017} use Linear Regression (LR) and Random Forest (RF) algorithms to detect and categorise anomalies in a Cloud data centre. 
%Before using these machine learning algorithms they use a feature selection scheme to reduce the number of features required to build the machine learning model. This has improved the performance of the algorithms in terms of anomaly detection accuracy. 
%this paper, we investigate both detecting and categorizing anomalies rather than just detecting, which is a common trend in the contemporary research works. We have used a popular publicly available dataset to build and test learning models for both detection and categorization of different attacks. To be precise, we have used two supervised machine learning techniques, namely linear regression (LR) and random forest (RF). 
%However, this system requires a lot of training data due to the nature of the algorithms.
%An Intelligent IDS for Private Cloud Environment has been proposed in~\cite{B.2015}). This approach which utilises a hardware module and a software application is based on previous history of intrusion traces which are given to the system during a training phase.
%In \cite{al2015applying} an anomaly intrusion detection model has been proposed to deal with attacks and security violations in Cloud environments. The detection model is utilising Hopefield Artificial Network and Simulating Annealing as aggregator which results in a detection rate of 93\% or less.
Gulenko et al.~\cite{ML_based_ids:2016} exploit various machine learning algorithms to detect anomalies in Cloud host machines. They use a combination of two types of data sets for evaluating the algorithms: ``normal" operation data and ``anomalous" data obtained via anomaly injection. They train the machine learning models offline and use them to detect the anomalies at runtime. 
%The results from~\cite{ML_based_ids:2016} indicate that machine learning algorithms are able to predict cloud anomalies with high accuracy. However, authors observed that the models trained for the host machines are affected due to ageing effects. Accuracy of the algorithms was degraded while evaluating them by separating the training and the testing data by time. 
The supervised learning algorithms used in Cloud anomaly detection as discussed above require training of the machine learning models with both ``normal" and ``anomalous" traces. 
%They usually collect the ``anomalous" traces from online repositories or generate them artificially.  
%These algorithms may fail to detect novel attacks, which are not recorded by the learning models or which have very different patterns than the learned attack patterns. 
These algorithms may fail to detect anomalies due to unknown attacks, traces of which are not recorded by the learning models or which have very different patterns than the learned ``anomalous" patterns.
To solve this problem researchers have proposed unsupervised learning algorithms which we discuss next. The unsupervised learning algorithms do not require labelled training data, i.e. they can build the learning models without the ``anomalous" traces.  
%whose attack pattern are very much different than learned attack patterns. We now discuss below how anomaly detection techniques can be used to overcome this limitation:
%Anomaly-based IDSs are more acceptable than the signature-based IDSs due to their strength in detecting unknown security attacks, which do not have any samples available for producing rules or finding signatures for signature-based IDSs. Moreover,  the most common cloud intrusions such as DDoS attacks, attacks on the hypervisor or on the virtual machine typically cause disruption on the cloud system's normal utilisation of either network, a computational resource, storage or a virtual machine's functionality. Such disruptions can be successfully captured by an anomaly-based IDS to identify them as a cause of cloud intrusion. 

\textit{(ii) Unsupervised learning algorithms.}
The authors in~\cite{automated-detection:2016} propose a mechanism for automatic anomaly detection and root cause analysis for Cloud data centres. They use an unsupervised K-Means clustering algorithm to identify the ``abnormal" system behaviour. 
%They use CPU utilisation and network traffic data for the analysis. 
%By performing an emulated DDoS attack on a cloud tested, they showed that their mechanism accurately identifies cloud anomalies and their causes.
UBL proposed in~\cite{UBL:2012} uses an unsupervised Self Organising Map (SOM) algorithm to predict unknown anomalies. SOM is computationally less expensive than K-Nearest Neighbour~\cite{knearest:2005}. 
%UBL requires only system-level metrics (CPU, memory, network, IOPS) to achieve black-box anomaly prediction. 
UBL predicts anomalies by identifying early deviations from ``normal" system behaviour. 
%Experimental results in~\cite{UBL:2012} show that UBL can accurately predict performance anomalies with sufficient lead time to prevent the anomalies.
\cite{density-based:2016} proposes a Cloud anomaly detection technique based on the concept of data density introduced by~\cite{density-based_ref1:2011}, which implements non-parametric Cauchy function~\cite{density-based_ref2:2010}. 
This technique computes the density recursively and therefore, it is memory-less and unsupervised.  
%The authors in~\cite{density-based:2016} evaluated the performance of the proposed technique based on the emulated dataset from a testbed, under different types and intensities of attacks. with high accuracy. 
%The evaluation results show that the proposed technique can achieve high accuracy of detection.
The authors in~\cite{EbAT:2010}, \cite{entorpy_based_detection_2:2014} measure the entropy of the system metrics such as CPU, memory, network, IOPS, etc., in order to identify Cloud anomalies. The entropy values indicate the dispersal or concentration of the metric distributions and they form the time series data for anomaly analysis.
%In~\cite{EbAT:2010}, the metrics are aggregated by entropy distributions across the cloud stack in order to form entropy time series. 
%EbAT, proposed in~\cite{EbAT:2010}, uses online tools like spike detection, signal processing and subspace methods to detect anomalies in the entropy time series. 
The approach proposed in~\cite{entorpy_based_detection_2:2014} identifies a Cloud security attack by observing whether the entropy variables obey normal distribution or not. They use Kolmogorov-Smirnov test (K-S test) to identify whether the entropy variables obey normal distribution. 
%When the Cloud data centre encounters an intrusion, the entropy variables do not obey normal distribution, which indicates an ``anomaly" due to attack. 
%Both ~\cite{EbAT:2010} and \cite{entorpy_based_detection_2:2014} can detect the cloud anomalies with high accuracy.
Recently, entropy has been used in various network anomaly detection tools~\cite{entorpy_based_detection_1:2005}, \cite{entorpy_based_detection_5:2014}, \cite{entorpy_based_detection_3:2017}, \cite{entorpy_based_detection_4:2017}. These tools firstly measure the entropy associated with the network traffic or network packet features (IP addresses and ports) and secondly they detect network attacks by observing the variation in the entropy values.
In our previous works~\cite{ladt:2015, ls-ladt:2016} we proposed a Lightweight Anomaly Detection Tool (LADT) which can detect anomalies on the hosting node level by using a correlation based algorithm. The algorithm utilises performance metrics on the hosting node level and the VM level to track disparities on the resource usage and detect host level attacks such as a Blue Pill attack~\cite{bluepill:2006}. However, this approach is not able to detect anomalies in the VM level which is the case for the current paper.

Although the unsupervised learning algorithms discussed above can detect Cloud anomalies due to unknown security attacks with high accuracy, they may generate false positives which arise mainly due to the workload spikes in a Cloud data centre.
%suffer from false alarm issues generated due to the wrong identification of the genuine Cloud workload spikes as ``anomalies". 
%Some algorithms~\cite{cloud-malware:2016}, \cite{automated-detection:2016}, \cite{UBL:2012} consider pre-processing the raw data using averaging in order to reduce false positives, but that is not sufficient. 
%The proposed anomaly detection system in this paper (RADS) uses One Class Classification (OCC) algorithm that is proposed by Hempstalk et al. in~\cite{OCC:2008}.
%and implements a window based data pre-processing approach which considers both the average and standard deviation of the raw data. RAIDS data pre-processing approach provides better accuracy and false alarm rate than the state-of- the-art average or entropy based data pre-processing approaches.
%RAIDS provides better accuracy and false alarm rate while using its data pre-processing approach, instead of using the state-of- the-art average or entropy based data pre-processing approaches.
%RAIDS uses the One Class Classification (OCC) algorithm that is proposed by Hempstalk et al. in~\cite{OCC:2008} to detect Cloud anomalies arising due to unknown security attacks. 
%In the literature, we find only the work in~\cite{cloud-malware:2016} that uses OCC algorithm for Cloud intrusion detection; specifically 
The authors in~\cite{cloud-malware:2016} propose a novel approach for Cloud malware detection using one class Support Vector Machine (SVM) algorithm. One class SVM is an extension of the traditional two-class SVM, which was proposed by Sch\"{o}lkopf et al. in~\cite{one_class_svm:1999}. 
Similar to the OCC algorithm~\cite{OCC:2008} that is used in this paper, one class SVM takes the unlabelled training data and produces a binary class based on the distribution of the training data. The binary class is composed of a known class, which is the ``normal" VM behaviour and a novel class, which is the unknown class representing the ``anomalous" instances. 
The work in this paper is different from that in~\cite{cloud-malware:2016} as this work focuses more on increasing the accuracy while reducing the false positives arising due to genuine Cloud workload spikes; whereas, \cite{cloud-malware:2016} focuses on reducing false positives arising due to VM live-migration.
%Some algorithms~\cite{cloud-malware:2016}, \cite{automated-detection:2016}, \cite{UBL:2012} consider pre-processing the raw data using averaging in order to reduce false positives, but that is not sufficient. 
%The proposed anomaly detection system in this paper (RADS) uses One Class Classification (OCC) algorithm that is proposed by Hempstalk et al. in~\cite{OCC:2008}.
%%and implements a window based data pre-processing approach which considers both the average and standard deviation of the raw data. RAIDS data pre-processing approach provides better accuracy and false alarm rate than the state-of- the-art average or entropy based data pre-processing approaches.
%%RAIDS provides better accuracy and false alarm rate while using its data pre-processing approach, instead of using the state-of- the-art average or entropy based data pre-processing approaches.
%%RAIDS uses the One Class Classification (OCC) algorithm that is proposed by Hempstalk et al. in~\cite{OCC:2008} to detect Cloud anomalies arising due to unknown security attacks. 
%In the literature, we find only the work in~\cite{cloud-malware:2016} that uses OCC algorithm for Cloud intrusion detection; specifically \cite{cloud-malware:2016} proposes a novel detection approach for Cloud malware detection using one class Support Vector Machine (SVM) algorithm. One class SVM is an extension of the traditional two-class SVM, which was proposed by Sch\"{o}lkopf et al. in~\cite{one_class_svm:1999}. Similar to the OCC algorithm that is proposed in~\cite{OCC:2008}, one class SVM takes the unlabelled training data and produces a binary class based on the distribution of the training data. The binary class is composed of a known class, which is the ``normal" VM behaviour and a novel class, which is the unknown class representing the ``anomalous" instances. The work in this paper is different from the work in~\cite{cloud-malware:2016} as this work focuses more on increasing the accuracy while reducing the false alarms arising due to genuine Cloud workload spikes and optimising the training time required for building the OCC models. Whereas, \cite{cloud-malware:2016} focuses on reducing false alarms arising due to VM live-migration.
%Our previous work in~\cite{ls-ladt:2016} verifies the detected anomalies considering the naturally occurring Cloud management activities such as VM migration, creation, suspend, terminate, etc. in order to reduce false alarms. 
% which was addressed in our previous work in~\cite{ls-ladt:2016} by verifying the detected anomalies considering the naturally occurring Cloud management activities such as VM migration, creation, suspend, terminate, resize etc. 

%Although the unsupervised learning algorithms discussed above can detect unknown Cloud security attacks with high accuracy, they may suffer from false alarm issues generated due to the wrong identification of the genuine Cloud workload spikes as ``anomalies". 
%%All these algorithms use system metrics such as CPU, memory, network, IOPS etc in their anomaly analysis and these metrics are prone to producing instantaneous spikes [REF]. 
%Some of these algorithms~\cite{cloud-malware:2016}, \cite{automated-detection:2016}, \cite{UBL:2012} consider pre-processing the raw data using averaging in order to solve the false alarm issues. 
%%However, considering some specific cloud application running scenarios, the proposed IDS in this paper shows better accuracy and lower false positive rate while using its unique data pre-processing technique instead of using the state-of-the-art average and entropy based techniques. 
%The proposed IDS in this paper (RAIDS) uses a combination of average and standard deviation in its data pre-processing phase and uses the One Class Classification (OCC) algorithm that is proposed by Hempstalk et al. in~\cite{OCC:2008}. RAIDS provides better accuracy and false alarm rate while using its data pre-processing approach, instead of using the state-of-the-art average or entropy based data pre-processing approaches. 
%In the literature, we find only the work in~\cite{cloud-malware:2016} that uses OCC algorithm for Cloud anomaly detection.
%\cite{cloud-malware:2016} proposes a novel detection approach for Cloud malware detection using one class Support Vector Machine (SVM) algorithm. One class SVM is an extension of the traditional two-class SVM, which was proposed by Sch\"{o}lkopf et al. in~\cite{one_class_svm:1999}. Similar to the OCC algorithm that is proposed in~\cite{OCC:2008}, one class SVM takes the unlabelled training data and produces a binary class based on the distribution of the training data. The binary class is composed of a known class, which is the ``normal" VM behaviour and a novel class, which is the unknown class representing the ``anomalous" instances. While detecting Cloud anomalies~\cite{cloud-malware:2016} focuses on an important pragmatic Cloud-oriented scenario, i.e. VM live-migration. Our previous work in~\cite{ls-ladt:2016} verifies the detected anomalies considering the naturally occurring Cloud management activities such as VM migration, creation, suspend, terminate, resize etc. 
%%analyses the console logs collected from the cloud data centre in order to assist the anomaly detector to verify whether an anomaly is caused by naturally occurring management activity such as VM migration, etc., or is indeed a true anomaly. 
%Our work in this paper is different from the work in~\cite{cloud-malware:2016} as we focus more on increasing the accuracy while reducing the false alarms of the IDS and optimising the training time required for building the OCC models.  

%\textit{(ii) Efficient and scalable learning algorithms: }
%To address the challenge of \textit{efficient and scalable analysis}, researchers in~\cite{cloud-malware:2016} and \cite{UBL:2012} implement the data monitoring and analysis for the intrusion detection in a distributed and decentralised way, where they execute the monitoring and the analysis of the data on each Cloud hosting node locally. On the other hand, researchers in~\cite{automated-detection:2016} propose Apache Kafka\footnote{https://kafka.apache.org} and Apache Spark\footnote{http://spark.apache.org} based data monitoring and analysis, where they deploy data collector agents on each VM which send the monitoring data to the central monitoring server. The monitoring server analyses the data in distributed way using Spark. However, the researchers do not adequately analyse the detection latency of their approaches, rather their focus remain in achieving the scalability of the IDSs.

%\textit{(i) One class classification algorithms: }
%\subsection {Unsupervised and semi-supervised learning algorithms for cloud IDSs}
%One class classification (OCC) is a form of classification, which can work only with a single class of data, without requiring a second class of data as it is the case in binary class classification.
%OCC algorithms have been used in anomaly detection for their capabilities in outlier/novelty detection and concept learning in scenarios where data from negative class is absent, poorly sampled or not defined well~\cite{OCC:2008}. In this work we propose RAIDS which utilises OCC models built from cpu and network utilisation data collected from the VMs on each hosting node. 
%\textit{(iii) Entropy based IDSs:} 
%\subsection {Use of statistical approaches to mitigate false alarm issues}
%  ~\cite{cloud-malware:2016}

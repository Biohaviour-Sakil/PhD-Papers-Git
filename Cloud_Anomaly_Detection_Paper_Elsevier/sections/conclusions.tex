\section{Conclusion}
\label{sec:conclusions}
\noindent Cloud computing services have seen significant growth in recent years. Such growth has attracted various cybersecurity attacks on Cloud data centres. Reports from various security experts have raised concerns regarding the potential damage and growth of the cybersecurity attacks in the Cloud. 
Researchers have proposed a number of anomaly detection techniques to deal with such attacks. However, there exists some challenges, specifically due to the unknown behaviour of the attacks and the occurrence of genuine Cloud workload spikes.
%According to reports from various security experts, Cloud services, specifically Infrastructure as a Service (IaaS) have seen massive growth in recent years. In spite of such growth, many users and organisations face barriers in adopting the Cloud due to its security concerns. Amongst the various Cloud security attacks, DDoS and backdoor channel attacks are growing sharply. Both DDoS and backdoor channel attacks significantly consume the system resources allocated to the hosted VMs in a Cloud data centre. Therefore, anomaly-based or behaviour-based intrusion detection systems (IDSs) are the ideal candidates to detect such attacks in Cloud. 
%Although existing research proposes a number of anomaly-based IDSs for Cloud, they encounter a number of challenges, specifically due to the unknown behaviour of the attacks and the occurrence of genuine Cloud workload spikes.
In this paper, we discuss these challenges and investigate the issues with the existing Cloud anomaly detection approaches. Then, we propose a Real-time Anomaly Detection System (RADS) which uses One Class Classification (OCC) algorithm and a window-based time series analysis to address the challenges. 
%RADS builds OCC model for each Cloud application running in the hosting node. The OCC model learns the ``normal" behavioural pattern in terms of the application's resource utilisation and flags an intrusion whenever the application's behavioural pattern deviates significantly from its ``normal" pattern.
%Specifically, RAIDS can detect Cloud security attacks such as DDoS and backdoor channel attacks. 
%RADS can operate in real-time, meaning that it can monitor the VMs running different Cloud applications in the Cloud data centre and detect cybersecurity attacks as they appear inside the monitored VMs.
%RADS can operate in real-time, meaning that it can monitor each VM hosted in the Cloud data centre in real-time and detect the attacks as they appear.

%\textcolor{red}{We evaluate the performance of RAIDS by performing both real-time and offline experiments. 
%The real-time experiments were performed in a lab-based Cloud data centre, which runs two representative Cloud applications (Graph Analytics and Media Streaming) from the CloudSuite workload collection, whereas the offline experiments were carried out on the real-world workload traces collected from a Cloud data centre named Bitbrains. 
%Evaluation results demonstrate that RAIDS can achieve 90-95\% accuracy (F1 score) while detecting the Cloud intrusions such as DDoS and backdoor channel attacks in real-time. The results further reveal that RAIDS experiences less number of false alarms while using the proposed data pre-processing approach instead of using the state-of-the-art average or entropy based approaches.
%We also evaluate the efficiency of RAIDS in performing the training and the testing in real-time in our lab-based Cloud data centre while hosting varying number of VMs (2-10 VMs). The evaluation results suggest that RAIDS can be used as a lightweight IDS for the Cloud data centres as it can perform the required training and the testing for the intrusion detection by consuming minimal computing resources and time. However, to attain a more realistic evaluation of the efficiency of RAIDS, we need to perform the experiment with more number of VMs hosted in our lab-based Cloud data centre.}
We evaluate the performance of RADS by running lab-based and real-world experiments.
The lab-based experiments were performed in an OpenStack based Cloud data centre, which hosts two representative Cloud applications (Graph Analytics and Media Streaming) collected from the CloudSuite workload collection, whereas the real-world experiments were carried out on the real-world workload traces collected from a Cloud data centre named Bitbrains.
Evaluation results demonstrate that RADS can achieve 90-95\% accuracy (F1 score) with a low false positive rate of 0-3\% while detecting DDoS and cryptomining attacks in real-time. The results further reveal that RADS experiences fewer false positives while using the proposed window-based time series analysis than when using state-of-the-art average or entropy based analysis.
We also evaluate the efficiency of RADS in performing the training and the testing in real-time in our lab-based Cloud data centre while hosting varying numbers of VMs (2-10 VMs). 
%The evaluation results suggest that RADS can be used as a lightweight anomaly detection system for Cloud data centres as it can perform the required training and the testing for the intrusion detection while consuming minimal computing resources and processing time. 
The evaluation results suggest that RADS can be used as a lightweight tool in terms of consuming minimal hosting node CPU and processing time in a Cloud data centre.
However, to attain a more realistic evaluation of the efficiency of RADS, we need to perform the experiment with a significantly greater number of VMs.
% hosted in our lab-based Cloud data centre.


\label{sec:conclusions}

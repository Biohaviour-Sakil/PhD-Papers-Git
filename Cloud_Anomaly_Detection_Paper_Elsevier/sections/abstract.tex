Cybersecurity attacks in Cloud data centres are increasing alongside the growth of the Cloud services market. 
%Cloud security attacks are increasing alongside the growth of the Cloud services market. 
Existing research proposes a number of anomaly detection systems for detecting such attacks. 
%These anomaly-based IDSs encounter a number of challenges, specifically due to the unknown behaviour of the attacks and the occurrence of genuine Cloud workload spikes, which must be distinguished from attacks. 
However, these systems encounter a number of challenges, specifically due to the unknown behaviour of the attacks and the occurrence of genuine Cloud workload spikes, which must be distinguished from attacks. 
In this paper, we discuss these challenges and investigate the issues with the existing Cloud anomaly detection approaches. Then, we propose a Real-time Anomaly Detection System (RADS) for Cloud data centres, which uses a one class classification algorithm and a window-based time series analysis to address the challenges. 
Specifically, RADS can detect VM-level anomalies occurring due to DDoS and cryptomining attacks.
%Specifically, RADS can detect DDoS and backdoor channel attacks in a Cloud data centre. 
%which can detect anoma- lies occurring due to DDoS and cryptomining attacks. 
%real-time with high accuracy and low false positive rates. 
%We evaluate the performance of RAIDS both in real-time and offline. The real-time evaluation was performed in a lab-based Cloud data centre whereas the offline evaluation was carried out with real-world Cloud workload traces.  
We evaluate the performance of RADS by running lab-based experiments and by using real-world Cloud workload traces. %The lab-based experiments were performed in an OpenStack based Cloud data centre.
%Real-world Cloud workload traces were collected from an online repository  
Evaluation results demonstrate that RADS can achieve 90-95\% accuracy with a low false positive rate of 0-3\%.
%The results further reveal that RADS experiences fewer false positives while using the proposed data pre-processing approach than when using state-of-the-art average or entropy based data pre-processing approaches.
The results further reveal that RADS experiences fewer false positives when using its window-based time series analysis in comparison to using state-of-the-art average or entropy based analysis.



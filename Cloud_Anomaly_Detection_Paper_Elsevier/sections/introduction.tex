\section{Introduction}
\label{sec:introduction}
\noindent The internet has become the dominant means of providing customer/citizen services for almost all business and government organisations. This is because of the advancement of Information and Communications Technology (ICT) mainly in the form of high speed internet connection, cloud computing services, and mobile devices. The growth of such internet-based or web-based services has attracted many cybersecurity attacks, which exploit the vulnerability of these services with evil intention. One of the major classes of cybersecurity attacks is the Distributed Denial of Service (DDoS) attack, which deny legitimate customers' access to the web-based services. Typically, DDoS attacks are launched by sending a redundant stream of network packets from a large number of compromised computer systems and/or mobile devices. 
%In most cases, the victims of such attacks are the prominent websites. 
The number of DDoS attacks and their volume have been increasing for the last few years. According to a report~\cite{coreo2} from Corero, in 2019, there was 35\% increase in DDoS attacks over 10Gbps as compared to 2018. This year, due to the COVID-19 pandemic people have turned to using the internet like never before - people are studying, working, shopping, and having fun online. This has shifted the gear of DDoS attacks - the new targets are websites of medical organisations, delivery services, educational, and gaming platforms. The impact of this can be observed in the latest DDoS attacks trend~\cite{kas} published by Kaspersky Lab: the number of DDoS attacks in Q1 2020 increased by 80\% against Q1 2019.  
%their customers experienced on average 8 DDoS attacks per day, which is 16\% more as compared to 2017. Also, the report reveals that in early 2018, the two largest-ever DDoS attacks hit the Github website (1.35 Tbps attack) and an unnamed US-based service provider (1.7 Tbps attack).
%The number of DDoS attacks and their volume have been increasing for the last few years. According to a report~\cite{coreo} from Corero, in 2018, their customers experienced on average 8 DDoS attacks per day, which is 16\% more as compared to 2017. Also, the report reveals that in early 2018, the two largest-ever DDoS attacks hit the Github website (1.35 Tbps attack) and an unnamed US-based service provider (1.7 Tbps attack).

To launch a successful attack, DDoS attacks must significantly consume network bandwidth. This results in an obvious change in the network traffic pattern or in the network packet information, which can be seen as network anomalies. Researchers have used various techniques such as~\cite{automated-detection:2016, UBL:2012, cloud-malware:2016} to detect network anomalies.
%, which include using machine learning and statistical approaches. 
%The anomaly detection systems proposed in~\cite{ml_based:2012, Pandeeswari2016, ML_based_ids:2016} use supervised machine learning algorithms. These algorithms require both the ``normal" and the ``anomalous" behaviour traces to build the learning models, which can detect the anomalies. 
%In a Cloud data centre, ``normal" traces can be prepared easily by monitoring the VMs' resource utilisation in the situation where the VMs are believed to be anomaly-free; however, ``anomalous" traces need to be generated artificially using emulation or collected from online repositories. 
%The algorithms may fail to detect anomalies arising due to unknown DDoS attacks, traces of which are not recorded by the learning models or which have very different patterns from the learned ``anomalous" patterns. 
%To solve this problem 
%To detect Researchers in~\cite{automated-detection:2016, UBL:2012, cloud-malware:2016} have proposed 
Specifically, they have used unsupervised learning and one class classification algorithms such as K-Means~\cite{automated-detection:2016}, Self Organising Map (SOM)~\cite{UBL:2012} , and one class Support Vector Machine (SVM)~\cite{cloud-malware:2016}. 
These algorithms first build the learning models by using normal network traffic or packet information and then use these models to identify anomalies by observing the deviation from the normal traffic pattern or packet information. 
Recently, entropy has been used in various network anomaly detection systems~\cite{entorpy_based_detection_5:2014}, \cite{entorpy_based_detection_3:2017}, \cite{entorpy_based_detection_4:2017}. These systems firstly measure the entropy associated with the network traffic or network packet features (IP addresses and ports), and secondly they detect network attacks by observing the anomalies in the entropy values.
%These algorithms build the learning models by using the ``normal" behaviour traces. The models can identify anomalies by observing the deviation in the ``normal" behaviour pattern. 
%and as a result, these algorithms can successfully detect zero-day or unknown attacks.
%These algorithms learn only from the ``normal" behaviour traces and do not use the ``anomalous" traces, and as a result, they can successfully detect zero-day or unknown attacks which impose significant deviation in the ``normal" behaviour pattern. 

%Although the techniques as mentioned above can detect network anomalies with high accuracy, they may exhibit false positives arising due to network traffic spikes. We can consider these spikes as "legitimate spikes" which do not follow the ``normal" traffic pattern and their values are significantly higher than the other values in the traffic data set. 
Although the techniques as mentioned above can detect network anomalies with high accuracy, they exhibit false positives, that is they generate anomaly alarms when there is no anomaly. Receiving such false alarms on a frequent basis is a major demerit of anomaly detection systems for a number of reasons: waste of operators' time as they engage in unnecessary investigations of the falsely raised alarms, unwanted interruption of services while the operator tries to mitigate the anomaly without realising that the alarm is false, etc. To investigate the cause for such false positives, we carried out a linear classification analysis of the network traffic collected from a real Cloud data centre trace~\cite{workloadCCGRID:2015} and a DDoS attack sample~\cite{caida}. 
%The reason for choosing the linear classification is that state-of-the-art anomaly-based techniques use such classification to detect DDoS attacks. 
%We performed this test analysis graphically by drawing the classification hyperplanes around the traffic data points. 
Our investigation suggested that traffic spikes play a significant role in generating false positives. We define these spikes as legitimate activity which does not follow the normal traffic pattern and their values are significantly higher than the other values in the traffic data. 
%As we investigated arising due to network traffic spikes. We can consider these spikes as "legitimate spikes" which do not follow the ``normal" traffic pattern and their values are significantly higher than the other values in the traffic data set. 
It is important to note that these spikes persist only for a momentary period of time and this differentiates them from the network anomalies (high network traffic) due to DDoS attacks, which persist for a relatively long period of time. We support this argument with evidence from real-world network traffic and DDoS attack. 
%Receiving false positive alarms on a frequent basis is a major demerit of anomaly detection systems for a number of reasons: waste of operators' time as they engage in unnecessary investigations of the falsely raised alarms, unwanted interruption of customers' services while the operator tries to mitigate the anomaly without realising that the alarm is false, etc. 
%This motivates a solution to remove false positives from the network anomaly detection systems. 
%Researchers in~\cite{automated-detection:2016}, \cite{UBL:2012}, \cite{cloud-malware:2016} consider window-based averaging on the raw data to reduce false positives. The works in~\cite{EbAT:2010} and \cite{entorpy_based_detection_2:2014} consider entropy-based anomaly detection which also reduces the number of false positives. However, these approaches may still generate false positives due to network traffic spikes, which we explain in the next section.
%Researchers in~\cite{automated-detection:2016}, \cite{UBL:2012}, \cite{cloud-malware:2016} use averaging on the raw data as a data pre-processing approach to reduce false positives. The works in~\cite{EbAT:2010} and \cite{entorpy_based_detection_2:2014} consider entropy analysis in their anomaly detection systems which also reduces the number of false positives. However, these approaches may still generate false positives in certain scenarios for certain use cases, which we explain in the next section.

To deal with the traffic spikes and to reduce the false positives, in this paper, we propose a linear regression based DDoS attack detection technique. The technique is based on the hypothesis that there is a positive correlation between average and standard deviation of the network throughput in a window-based time series, and this correlation is affected due to DDoS attacks. The hypothesis is supported by the findings in~\cite{variance}.
%ADDoSS works in two phases, the training phase and the detection phase. 
The proposed technique works in two phases: training and detection. During the training phase, it builds a linear regression model using the average and standard deviation values of the network throughput. During the detection phase, for a particular detection window, it firstly uses the linear regression model to predict the standard deviation of the network throughput for the given (measured) average value and secondly, it calculates the difference between the predicted and the measured standard deviation values in order to identify whether there is any anomaly in the detection window. 

Specifically, we make the following contributions in this paper:
\begin{enumerate}[{(1)}]
%\item RAIDS provides an algorithm based on probabilistic classification for detecting anomalies in cloud applications. The accuracy of the algorithm is examined using different virtual machines running various services from the CloudSuite workload collection. The proposed algorithm utilises probability distribution analysis on the raw data (??) to detect anomalies and minimise false positives.
%\item \textcolor{red}{We propose a new intrusion detection algorithm for Cloud that provides high accuracy and low false positives in detecting Cloud security attacks such as DDoS and backdoor channel attacks.}
\item We propose a linear regression based technique for detecting DDoS attacks with high accuracy and crucially low false positive rate. 
\item We evaluate the performance of the proposed technique by running experiments on real-world network traffic (collected from a Cloud data centre named Bitbrains~\cite{bitbrains} and CAIDA DDoS attack 2007 dataset~\cite{caida}. 
\item We compare the proposed technique against average and entropy based one class classification (OCC)~\cite{OCC:2008} techniques, which represent the state-of-the-art linear classification techniques to detect DDoS attacks. 
\end{enumerate}

%Evaluation results demonstrate that the proposed linear regression based technique reduces the false positive rates significantly while maintaining the accuracy of attack detection. The remainder of the paper is organised as follows. Section~\ref{sec:problem_definition} describes the problems with the existing approaches in DDoS attack detection. Section~\ref{sec:related_work} presents related work in network anomaly detection.
Evaluation results demonstrate that the proposed linear regression based technique reduces the false positive rate by 95\% and 99\% when compared against average and entropy based OCC techniques, respectively, while maintaining the accuracy of attack detection (92\%, which is 18\% and 114\% better than average and entropy based OCC techniques, respectively). 

The remainder of the paper is organised as follows. Section~\ref{sec:preliminary_investigation} investigates the issues with the existing DDoS attack detection techniques. 
Section~\ref{sec:proposed_technique} formulates the hypothesis and proposes the DDoS attack detection technique. 
Section~\ref{sec:evaluation} presents experimental results and discusses them. 
Section~\ref{sec:related_work} discusses the related work in network anomaly detection.
Finally, Section~\ref{sec:conclusions} concludes the paper. 
